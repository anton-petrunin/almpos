\documentclass{amsart}
\usepackage{almpos}

\usepackage[colorlinks=true,
citecolor=black,
linkcolor=black,
anchorcolor=black,
filecolor=black,
menucolor=black,
urlcolor=black,
pdftitle={On torsion in center conjecture},
pdfsubject={Differential geometry},
pdfauthor={Vitali Kapovitch, Anton Petrunin and Wilderich Tuschmann}
]{hyperref}

\begin{document}
\title[On torsion in center conjecture]
{On torsion in center conjecture:\\
{\small An appendix for ``Nilpotency, almost nonnegative curvature, and gradient flow''}}

\thanks{\it 2000 AMS Mathematics Subject Classification:\rm\
53C20. Keywords: nonnegative curvature, nilpotent}\rm
\author{Vitali Kapovitch}
\address{Vitali Kapovitch\\Department of Mathematics\\University of Toronto\\
Toronto, Ontario, M5S 2E4, Canada}\email{vtk@math.toronto.edu}
\author{Anton Petrunin }\address{Anton Petrunin\\ Department of Mathematics\\ Pennsylvania State University\\
University Park, State College, PA 16802
}\email{petrunin@math.psu.edu}
\author{Wilderich Tuschmann}\address{Wilderich Tuschmann
\\Mathematisches Seminar
\\Christian-Albrechts--Universit\"at zu Kiel\\ Ludewig-Meyn--Stra\ss{}e 4-8 \\ D-24118 Kiel,
Germany}\email{tusch@math.uni-kiel.de}

\maketitle
\begin{abstract}
We report on our attempt to prove that
any almost nonnegatively curved $m$-dimesional manifold $M$ 
admits a finite cover $\tilde M$ with number of folds bounded in terms of $m$
such that the torsion of the fundamental group $\pi_1 \tilde M$ lies in its center.

A source of examples of almost nonnegatively curved manifold is provided by the total spaces of towers of fiber bundles of certain type. We prove the concusion of the torsion in center conjecture for these examples.
\end{abstract}


\section{Introduction}

In \cite{KPT} we formulated the following conjecture.

\begin{mconj}\label{con:tor}
There is $C=C(m)$ such that if $M^m$ is almost nonnegatively curved then there is a nilpotent subgroup $N\subset \pi_1M$ of index  at most $C$ whose torsion is contained in its center.
\end{mconj}

Note that it generalizes the following conjecture formulated by Fukaya and Yamaguchi in \cite{FY}.

\begin{conj}[Fukaya--Yamaguchi]\label{con:c-ab}
The fundamental group of a nonnegatively curved  $m$-manifold is $C(m)$-abelian;
that is, there is $C=C(m)$ such that if $M^m$ is nonnegatively curved then there is abelian subgroup $A\subset \pi_1M$ of index  at most $C$.
\end{conj}

Indeed, Conjecture ~\ref{con:tor} immideately implies that the fundamental
groups of closed positively curved $m$-manifolds are $C(m)$-abelian.
Further, if $\sec(M)\ge0$, then the universal cover $\tilde M$ of $M$ is isometric to the product $\mathbb{R}^n\times K$, where $K$ is a compact Riemannian manifold and the $\pi_1M$ action  on $\mathbb{R}^n\times K$ is diagonal.
It follows from \cite[Cor. 6.3]{wilking} that one can deform the metric on $M$ so that its universal cover is still isometric to $\mathbb{R}^n\times K$ and the induced action on $K$ is finite.
By passing, as in the proof of \cite[Corollary 4.6.1]{KPT}, to the induced action on the frame bundle of $K$, one reduces the statement to Conjecture~\ref{con:tor}.
The described  proof for nonnegative curvature was suggested to us by Burkhard Wilking.

The natural approach to the main conjecture would be studying successive blow-ups of the
collapsing sequence $M_n$ as done in \cite[Section 4.3]{KPT}.
In this note we show that the Conjectures~\ref{con:tor} is true if all the spaces $A_i$ which appear in the construction in the construction are closed Riemannian manifolds.
This statement can be reformulated using purely topological language as below.
The reduction is done by using Yamaguchi's fibration theorem \cite{Yam};
see the sketch in the Section~\ref{sec:reduction}.

\begin{thm}\label{thm:smooth}
Let $F_1,F_2,\dots,F_n$ be an array of compact manifolds 
such that each $F_i$ is either $\mathbb{S}^{1}$ or is simply connected. 
Assume $E$ is the total space of a tower of fiber bundles over a point
$$E=E_n\buildrel {F_n}\over \longrightarrow E_{n-1}\buildrel {F_{n-1}}\over\longrightarrow\dots\buildrel {F_1}\over\longrightarrow E_0=\{pt\}$$
and each of the bundles $E_k\buildrel {F_k}\over \longrightarrow E_{k-1}$ is homotopically trivial over the $1$-skeleton. 
Then the fundamental group $\pi_1E$ contains a nilpotent subgroup $N$ such that
$$[\pi_1E:N]\le \mathrm{Const}(F_1,F_2,\dots,F_n)\quad\text{and}\quad\mathrm{Tor}(N)\subset \mathrm{Z}(N),$$
where $\mathrm{Tor}(N)$ and $\mathrm{Z}(N)$ denote the torsion and the center of $N$ correspondingly.
\end{thm}

Mathematically speaking we just prove this statement;
the rest of the paper is the motivation.
The proof we found is surprisingly complicated.
For several years we were trying to extend it to the singular fibrations, and prove this way the main conjecture.
Finally we decided to publish the result as it is in hope that someone else will find a way out.

Please write us if you know a proof of \ref{thm:smooth} based on a different idea,
such a proof might lead to the solution of the main conjecture.

\section{Reduction}\label{sec:reduction}

In this section we sketch a reduction of the main conjecture (\ref{con:tor}) to the main theorem (\ref{thm:smooth}) in the case if all the partial limits of all blow-ups in the construction given in \cite[Subsection 4.3]{KPT} are smooth. 
First we need to formulate the statement precisely. 

Let $M$ be almost nonnegatively curved manifold.
Equip $M$ with a sequence of metrics $g_n$ such that $\sec(g_n)>\tfrac1n$ and $\diam (M,g_n)\to 0$ and $n\to\infty$.
Applying the blow-up construction as in \cite[Subsection 4.3]{KPT},
for any choices of points $\bar p_n\in M_n=(M,g_n)$
we get a sequence of Alexandrov spaces $ A_1,\dots, A_\ell$.

Now assume that for all choices of the sequence $\bar p_n\in M_n$
the Alexandrov spaces $A_1,\dots, A_\ell$ are Riemannian.
(This is assumption is totally unjustified.)

In this case after passing to a subsequence of $M_n$,
each $M_n$ is homeomorphic to the total space of a tower of fiber bundles
$$M_n=E_\ell\buildrel {A_\ell}\over \longrightarrow E_{n-1}\buildrel {A_{\ell-1}}\over\longrightarrow\dots\buildrel {A_1}\over\longrightarrow E_0=\{pt\}.$$

Indeed, after passing to a a subsequence we know that the rescaling $\tfrac1{\theta_{1,n}}\cdot M_n$ converge to $A_1$.
By Yamaguchi fibration theorem each $M_n$ is a fiber bundle over $A_1$, denote its fiber by $M_{n,1}$.
Each fiber $M_{n,1}$ is close to the space $A_2$ for the appropriate choice of the sequence $\bar p_n\in M_n$
and it fibers over $A_2$ with fiber $M_{n,2}$.
It follows that we can build first two lower levels of the tower
$$M_n\buildrel {M_{n,2}}\over \longrightarrow A_1=E_1\buildrel {A_1}\over\longrightarrow E_0=\{pt\}.$$
Continuing this way we get the needed tower.

Each $A_i$ has nonnegative curvature.
In particular, after passing to a finite cover $A_i$ is a total space a fiber bundles over torus with the simply connected fibers.
Moreover the induced $A_i$-bundle over any closed loop $\gamma$ on $E_{i-1}$ is homotopic to the cylinder  $A_i\times[0,1]$ with the top $A_i\times 1$ glued to the bottom $A_i\times 0$ along an isometry.
This can be proved along the same lines as the corollary 3.1.2. 

It follows that there is a constant $k_i$ which depends only on the $A_i$ such that  
after passing to at most $k_i$-folded  cover of $E_{i-1}$, we can assume that $A_i$-bundle over $E_{i-1}$ admits a tivialization on 1-skeleton.

Indeed, applying the gradient flow estimates as in ??? we get that $A_i$ and $A_i'$
are the spaces obtained for different choice of the sequences $\bar p_n\in M_n$
then there are homotopy equivalences $A_i\to A_i'$ and $A_i'\to A_i$ with uniform bound on the Lipschitz constant.


\begin{thm}
Let $M_n$ be a sequence of Riemannian manifolds such that $\sec(M_n)>\tfrac1n$ and $\diam M_n\to 0$ and $n\to\infty$.

Assume that for any choices of points $\bar p_n$ each of the Alexandrov spaces $A_0,A_1,\z\dots,A_\ell$ in the blow-up construction from \cite[Subsection 4.3]{KPT} are Riemannian manifold.
(This is assumption is totally unjustified.)

Then after passing to a subsequence of $M_n$,
each $M_n$ is homeomorphic to the total space of a tower of fiber bundles
$$M_n=E_n\buildrel {F_n}\over \longrightarrow E_{n-1}\buildrel {F_{n-1}}\over\longrightarrow\dots\buildrel {F_1}\over\longrightarrow E_0=\{pt\}$$
for a fixed array $F_1,F_2,\dots,F_n$  of compact manifolds 
such that each $F_i$ is either $\mathbb{S}^{1}$ or is simply connected.
Moreover, each of the bundles $E_k\buildrel {F_k}\over \longrightarrow E_{k-1}$ is homotopically trivial over the $1$-skeleton.  
\end{thm}

\begin{proof}
Recall that after passing to a a subsequence we know that the rescaling $\tfrac1{\theta_{1,n}}\cdot M_n$ converge to $A_1$.
By Yamaguchi fibration theorem each $M_n$ is a fiber bundle over $A_1$,
denote its fiber by $M_{n,1}$.

The manifolds $M_{n,1}$ also collapse to a point and they with curvature bounded below in a generalized sense.
Namely, if $k_1=\dim A_1$ then there is a metric on $M_{n,1}\times\mathbb{R}^{k_1}$ such that (1) ??? and (2) ???.
The Yamaguchi fibration theorem admits a straightforward generalization to the described type of collapsing.
This way we get a fibration of $M_{n,1}$ over $A_2$ with the fiber $M_{n,2}$.

Continuing this way we get a sequence 
$$\{p_n\}=M_{n,\ell}\subset \cdots \subset M_{n,2}\subset M_{n,1}=M_n.$$
One can build a tower of fiber bundles which build from the top to the butom.
Namely 
$$M_{n,i}\buildrel{M_{n,i+1}}\over\longrightarrow A_i$$

Let us show that the tower can be also build from below.
That is,
$$M_n=E_\ell'\buildrel {A_\ell}\over \longrightarrow E_{\ell-1}'\buildrel {A_{\ell-1}}\over\longrightarrow\dots\buildrel {A_1}\over\longrightarrow A_0=\{pt\}.$$

...

Consider a loop $\gamma$ in $E'_i$.
Lift it to a homotopy of maps 
Let $f_t\colon A_{i+1}\to E'_{i+1}$.
From ???, it foloows that the map $f_1\colon A_{i+1}\to A_{i+1}$ is homotopic to an isometry.
In particular, after passing to a finite cover with number of fibers depending on geometry of $A_i$,
we can assume that $f_1$ is homotopic to the identity map.
It follows that we can assume that all the fiber bundles in the tower admit a trivialization over 1-skeleton.

Our next aim is to split the constructed tower in a finer tower, exchanging each $A_i$-bundle by a tower of $\mathbb{S}^1$-bundles and ???.


...



\end{proof}


\small
\bibliographystyle{alpha}
%\bibliography{kpt}

%\begin{comment}
\begin{thebibliography}{HMR75}
\bibitem[DDK]{DDK}{Dror, E. and Dwyer, W. G. and Kan, D. M.},
 \textit{Self-homotopy equivalences of virtually nilpotent spaces},
{Comment. Math. Helv.},
{56},
{1981},
{4},
pp. {599--614},

\bibitem[FY]{FY}
K.~Fukaya and T.~Yamaguchi.
\newblock The fundamental groups of almost nonnegatively curved manifolds.
\newblock {\em Ann. of Math. (2)}, 136(2):253--333, 1992.


\bibitem[KPT]{KPT} Kapovitch, Vitali, Anton Petrunin, and Wilderich Tuschmann. 
\newblock Nilpotency, almost nonnegative curvature, and the gradient flow on Alexandrov spaces. 
\newblock Annals of Mathematics (2010), 343--373.

\bibitem[W]{wilking}
Wilking, Burkhard. \newblock On fundamental groups of manifolds of nonnegative curvature. \newblock  Differential Geom. Appl.  13  (2000),  no. 2, 129--165.

\bibitem[Y]{Yam}
T.~Yamaguchi.
\newblock Collapsing and pinching under a lower curvature bound.
\newblock {\em Ann. of Math.}, 133:317--357, 1991.
\end{thebibliography}


\end{document}